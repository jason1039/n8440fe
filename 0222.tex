`\subsection{父母之间}

「……给他们添麻烦了啊」\\

确认树和千岁离去后,原本在远望的大辉苦笑着往这边靠近。

周也感到非常尴尬和不好意思,但他不便对别人家的问题刨根问底,只能无所作为地目送两人离去。\\

走到附近后,志保子也注意到了大辉,便和真昼一起走了过来。\\

「啊,是刚刚见过的树的爸爸」

「久仰久仰,我家儿子承蒙你们关照了」

「不不,我才是……」\\

先是常见的相互寒暄,然后周的父母和大辉相互报上了姓名。看着这个场面,周感到心里不畅快。\\

「……啊,那个,大辉叔叔,刚才那是……」

「这些事情都在预料之中。我对她太严格,树要躲着我也很正常」\\

与其说是悲伤,大辉更像是死了心,淡然地接受了。修斗和志保子似乎也明显察觉出大辉和他儿子树的\ruby{女朋友}{千岁}关系不好,有点担心地垂下眉头,他们大概是想起来周以前闲聊时有讲过:有个朋友谈恋爱没有得到家长的认可,正感到为难呢。\\

大辉看样子并不介意父母的神情。他好像要回忆刚才发生的事情似的看向斜上方,然后轻轻一笑。\\

「说起来,椎名和藤宫君的爸妈关系真好啊,我看着都很吃惊」

「多谢夸奖」

「她可是未来的女儿,再说还那么乖,乖得让人想疼她」\\

修斗和志保子性格如此,周与真昼的交往也获得了父母的认同,将来真昼就是女儿,关系好也是自然。不过,这话说出来对大辉而言犹如挖苦,所以周没好意思讲……志保子却毫不介意,坦坦荡荡地直说了。\\

周觉得这恐怕是有意为之。志保子似乎有她的想法,修斗也无意阻止。\\

听到志保子断言说自己不带任何恶意、纯粹地中意真昼,真昼害羞起来,大辉则睁大眼睛不知所措,过了会儿才露出苦笑。\\

「也对,既然是她,两位应该也不会有什么不满意」

「嗯,她可是儿子挑的人。小真昼长得端端正正的,我们见到后也觉得,可以放心把周交给她」\\

在父母眼里是把周交给真昼而不是反过来,这让周有点不服气,但他的确是在受着照顾,也不好有什么意见。\\

「好羡慕你们啊,我家蠢儿子就不行」

「你不相信自己的儿子吗?」

「他啊,可比不上你们家儿子那么好。他还小,不懂事」

「哦,我觉得不是吧?我听周说他心地善良,也很会关心人」

「唔……」\\

大辉变得吞吞吐吐,而志保子面露沉静的微笑。

或许是同为家长,有什么感触吧,平时志保子不会追问这么多,这次她却毫不客气。

这一行动的关键原因,应该是看到了树护着女朋友逃离父亲的样子吧。\\

「我知道为人父母会对挑选的对象有些想法……但男孩子很快就会萌生出自立心理,压制太多会有反抗的。好不容易把孩子养得这么好,应该相信他看人的眼光,默默关注,我觉得这也是大人该做的」\\

志保子说完便向大辉微笑,面对这笑容,大辉摆出了吃着苦瓜般的面孔。

这不像是出于厌恶,反倒像是被戳到痛处导致的。\\

见志保子不再打算说下去,修斗露出淡淡的苦笑。\\

「好啦,虽然我们才刚认识,不太适合讲大道理……但是,既然孩子没有明显地误入歧途,他又打算走自己选择的路,那么即使制止,孩子也不会接受的」\\

总结完后,修斗便和志保子一样面带微笑地看着大辉。周挠挠脸,轻轻叹了口气。\\

周觉得并不该由自己插嘴,然而他却理解,不论说这是优点还是缺点,大辉至少是个顽固的人。周也知道,从父母的角度和自身的角度看待事物是有区别的。

既然大辉明白千岁不是坏人,那么剩下来的就是认识和要求的差异。\\

「大辉叔叔,请让我也说一句。那个,大辉叔叔……可能不太喜欢千岁……但她绝对不是个没出息的人,最近她还在烦恼想得到你的认可,也有在努力着。我不是说一定要接受她……但请从正面看看她吧」\\

只是大辉要求的标准太高,千岁自身并没有那么不行:她脑子没有极端地不好使;关键时候能够察言观色;也懂得关心人。

硬要说的话,不过是理想不同而已,周不希望大辉全盘否定。\\

周这番犹犹豫豫的话让大辉稍微吃了一惊,他尴尬地移开视线。\\

「……我也想要妥善处理。不过就算这样,她最好能再努力点。既然要肩负起我们家的名字,至少也要拥有配得上的器量」

「说的是,我会转告她的」\\

听到这含有「视情况妥协」之意的说辞,周暗自耸肩,并由于前进了一小步而舒了口气。
