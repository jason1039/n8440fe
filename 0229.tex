\subsection{回家与迎接}

唱完卡拉OK,周和树一行人分开,拖着有些疲累的身子回家了。

由于事先答应过,修斗和志保子已经到了周的家里,对周和真昼笑脸相迎。\\

「欢迎回来。不和朋友继续庆功了吗?」

「怕他们弄得太晚。就算我在,我也不想让真昼那么晚还到处乱转悠;而且晚饭应该都准备过了」\\

自己过着独居生活,母亲还来迎接,不免有些不自然;不过以前在老家时一直都是这样,也有怀念的感觉。

不知为何,真昼好像适应了似的,接受了这个状况,这或许得感谢暑假那件事情。看到志保子,她就变得喜笑颜开,只是这样,便足以让父母不虚此行了。\\

真昼带着平静的表情与志保子对话时,周走过她的旁边,前去自己房间换衣服。真昼是先换好衣服再到周家里来的,周进了家门后,她就和往常一样脱下鞋,跟着志保子去了客厅。\\

由于没打算仔细挑选,周就随便从衣柜里找了一件,换上后前去客厅。客厅里不见真昼,一丛亚麻色摇摆在厨房那边。\\

「说起来,妈妈你们晚饭怎么解决?」

「我们是外面吃完回来的,毕竟来得也挺仓促。我也告诉过小真昼了」

「其实我们本来想住酒店,好不打扰你们的」

「谢谢操心嘞,反正我今天住真昼家里」\\

之后还有一项活动:住在真昼家里——对周来说,这件事没准比文化节还重要;再说平时就是两人独处,没人打扰是常态。\\

听到住家里这个说法,志保子一下来了劲。\\

「对了对了,白天你也是这么说的,今天你要去过夜吗?」

「……就是这么回事,妈妈你们就在家里随意睡吧」

「哎呀哎呀,嘿嘿」

「……这笑的是什么意思?」

「没什么,就是我猜小真昼会期待的」

「怎么可能,不会的。还有,不要多管儿子谈恋爱」\\

哪怕是过夜,如果要对真昼做什么,想必她会更加紧张。尽管没有表现出来,但目前周比真昼还要紧张。周压根没进过女生的房间,更何况是女朋友的。\\

再说,这种话也不可能当着家长的面讲,周便无视志保子的询问,转而看向修斗。修斗则是笑嘻嘻地看着周,似乎并无劝告或是追问的意思,只是发表着「关系不错呀」这般淡泊至极的感想。\\

「年轻人嘛,不要放飞自我就好,倒是没想到周交了这么多朋友,不错哟」

「你把我当什么了」

「呵呵。要不是发自内心地信赖,周是不会融入进去的。看样子关系好的人还不少,那我就放心了」\\

周告诉过父母自己交到了一些要好的人,不过在亲眼看见之前,他们似乎还是有点担心。\\

「没怎么听你说过优太君,哪知道你有个那么帅气又心地善良的朋友」

「我自己也觉得挺神奇的。他的确是个好人」

「我觉得那是因为周君人好,物以类聚吧」\\

真昼在厨房似乎听到了谈话,切菜声中响起她的声音。

说到人好,周也没什么感觉,但合得来这点倒是没错。门胁是个夺人眼球的男生,但他性格上不喜显眼,而是温厚稳重,可能和喜爱安静环境的周性情相投吧。\\

「要这么说,我和树也是一类了」

「你们都很为朋友着想。你关心赤泽家里的情况,还想出力帮他改善不是吗」

「那肯定啦,关系好点总没什么不好的」\\

周见证了父母的恩爱,两人甚至被揶揄为一对鸳鸯;他也切身体会到父母对孩子的爱。周的家庭较之其他人应该是更加和睦的。\\

理所当然似的经历过这一切后,周虽然不想强加于人,但不禁希望树的家庭能以某种方式和解。

树的情况并不像真昼父母那样无法挽回,只要双方能够相互认可,和解也不无可能。\\

「赤泽也说,你嘴上不坦率,他也看得出你在关心他」

「下次我去叫他别多嘴」

「说的就是你这种地方」\\

一声嬉笑响起,周皱起了眉头。不知是看到周的表情还是听了真昼的话,父母也微微笑了出来,表情笑眯眯的。周尴尬地躲开所有人的眼神,一屁股坐到沙发上。

周装作不知,却引来了更多笑声,于是他嘟哝一句「有完没完」然后把心思集中到放着综艺节目的电视上。
