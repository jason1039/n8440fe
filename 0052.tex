\subsection{情人节的次日}

「藤宫,昨天多谢了」\\

第二天,周来到学校后,因为门胁过于自然的搭话而不由得僵住了。\\

虽说昨天稍微有一点来往,但周没想到他会特意为了这点小事而跑来道谢。\\

门胁的表情像是老好人一样明朗,和他被女生围住的时候并不一样。被他笑着搭话的周也收到了旁边若有若无的视线,感到十分难受。

周本来就不喜欢受人瞩目,面对这种充满好奇的视线还是会感到有些胆怯。\\

「啊,那点事情不用在意啦。看你也挺不容易的」

「算是吧……」\\

门胁露出了仿佛在看远方的眼神,周也同情地说道「果然受欢迎的男人很不好受啊」。\\

门胁本人自知受到欢迎,却并不为此而骄傲。正因如此,他才会受到周围人们的喜欢,并且嫉妒他的男生们也不会真的讨厌他。

或许,为这点小事特意道谢的守规矩的品格,说不定也是他得到其他人喜欢的原因。\\

「总之还是帮大忙了。想来道个谢」

「没事的啦,有困难的时候互相帮助嘛」\\

周也不是为了卖人情而帮他,所做的也并不是那么值得被感谢的事。

周轻轻笑着说不要在意之后,门胁也稍微露出了安心的笑容。\\

面对他发自内心的笑容,周围的女生们顿时喧闹了起来。周只得苦笑着感叹,这个笑容应该在面对女生的时候用啊。\\

\vspace{2\baselineskip}

「你和优太发生了什么吗」\\

门胁离开之后,刚刚似乎一直在旁观的树前来搭话了。\\

「优太」是门胁的名字。树和班上所有人关系都比较好,给人以待人和蔼、能炒热气氛的感觉,当然和门胁也有着一定的来往。

周有时也会感叹、困惑,这样的男人居然愿意当自己的朋友。\\

「呃,因为他收了太多巧克力,有些手足无措,我就把自己存着的购物袋给他了而已……」

「啊。收到的比他预计的还要多是吧。结果最后出了岔子啊」\\

树当时也在旁边看着那一大堆的巧克力和女生的好意。听到周的解释之后,他理解地露出了带有同情的苦笑。\\

那时,两个人的感想就是,有那么多的话带回去肯定很辛苦。因此,周给他帮忙也不是什么不可思议的事。

周倒是觉得,自己只是帮了个小忙,并不需要什么道谢。\\

「就只是这样了,也没做什么了不起的事」

「该说是像你的风格吗……不过,常备塑料袋啥的……怎么感觉你像辛苦的家庭主妇一样,特别是看你拿手机看超市广告的时候」

「我是男的啊。不过,应该是受了某人的影响吧……」\\

该说都是真昼害的,还是该说托她的福呢。\\

由于伙食费两个人各出一半,所以周为了尽可能节省,有时会浏览网上的广告,有时会向真昼提议去做广告里的便宜商品做得出的东西。在树的眼里,这样做就显得更加像个为家庭而奔波的人一样了吧。

或许,周所做的事情,比起一般的一家之主,反倒远远更像是主妇干的事。虽说料理全是交给真昼就是了。\\

「有个顾家的搭档真好啊」

「才不是什么搭档啊……千岁呢」

「小千?她啊,嗯,只要不把奇思妙想付诸行动的话,应该……也不是做不到吧」

「你意思是她会不乱搞?」

「……这一点也很可爱对吧?」

「喂别闪开视线啊」\\

往好里说往坏里说,千岁都是个随性而喜欢寻求刺激的人。

普通做的话,她也能做到一般女高中生等级的家务,但她要是起了玩心或者心情有变就会搞出很多事情。\\

「不过,她说结婚之后应该会老实点」

「要让你爸答应得花多久啊……」\\

树的父亲对交往管得很严,这在当下很少见。他不待见千岁,对当前两人以结婚为前提的交往有所不满。

千岁的父母倒是一直很欢迎树。周还曾经有点惊讶「一般都是反过来的吧……」\\

「我长大后会慢慢说服我爸的啦,就问他不想看孙子吗」\\

树做作地耸了耸肩膀,但眼中写满了认真,表示即便发生争吵,这件事上也决不会听父亲的话。

从平时的表现也看得出,他对千岁的爱很深。周觉得树从高中就开始考虑结婚很了不起,同时决定给他加油。\\

「……反正你爸放弃之前估计你也不会让步的,加油吧」

「嗯。你也加油」

「加油什么」

「和那个人……对吧?」

「……我和她又不是那种关系」\\

「别瞎猜」周说着把脸别开之后,旁边就传来了树明朗而愉快的笑声。

\psline

※既没有门胁君路线,他也不是真昼的情敌(看着感想栏)
