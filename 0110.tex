\subsection{笑着生气的天使大人}

「藤宫,这是怎么一回事」\\

结束了上午的比赛回到教室后,周果然陷入了受到班级男生质问的困境。\\

真昼既是高岭之花,也是受到憧憬的对象。那样的真昼,在众目睽睽之下将周作为重要的人借走了。周明白男生为此而心里不平,但突然被如此逼问,周也只会感到困惑。\\

「就、就算你们这么问」

「为什么你会是椎名的……重要的人!」

「话说,什么时候开始的!」

「你们没有接触吧!?一起吃午饭也没多久吧!?」

「到底是哪里!椎名到底觉得你哪里好了!」

「完全无法理解!」\\

质问的话语一句接一句,让周已经快要放弃回答了。\\

老实说,尽管他预料到会被质问,但男生的逼问还是超过了周的预期。他甚至连吃午饭的时间都没有。\\

当然,对这件事有反应的并非只有男生。女生们没有加入质问的行列,不过她们也向周投去了视线,有的像在打量他,有的看上去很愉快,有的还透出一种安心。\\

这恐怕是由于真昼这个女生最大的竞争对手喜欢着周。

打量的目光包含的意思则是「那个真昼所思慕的是怎样的人」。\\

班级的视线集中在周的身上,使他早早陷入了疲劳。\\

顺带一提,树和门胁在离男生们稍微远一些的地方,「啊啊」地苦笑着。千岁看着这边,表情带着微妙的兴奋。周很希望这些人能快来救场。\\

「请不要过分欺负他」\\

最先向周伸出援助之手的,正是事件中心的另一个人——真昼。

真昼可能是去买运动饮料了,因此晚了一步回到教室。在她的手上握有运动饮料的瓶子。由于天气稍微热了起来,瓶子上还沾着汗水。\\

与周对上目光后,真昼露出了柔和的笑容。

拜其所赐,男生们向周投去杀气,让他感到十二分的压力。\\

「明明是中午却吃不了午饭,周君正在困扰着哦?」\\

亲近的人之间才会直呼名字。真昼似乎是丝毫不想再隐藏她对周的称呼了。

即使受到男生女生的关注,真昼也毫不介意,这让一名男生无法忍耐,走到了真昼前面……他就是刚刚对周步步紧逼的那个人。\\

周围的人察觉到他要作为代表询问大家关心的事,主动给他让出一条道路。目前,对周的质问也已停止。\\

「椎名!你说藤宫是重要的人,是说……」

「没错,周君是我重要的人」\\

真昼果断地肯定后,脸上浮现与往常无异的微笑。\\

那名男生看到那天使大人的天衣无缝的微笑,畏缩了一瞬间;但或许是由于周围视线的支持,他尽管少了点气势,却还是继续问了下去。\\

「那、那个是……恋、恋人的意思吗」

「就算是那样,你想要对我说些什么?」

「不,我的意思是……假如,是恋人的话……为什么会是藤宫那种家伙」

「藤宫那种家伙?」

「啊不,那个,那么不起眼的藤宫,和椎名交往,是不是感觉不太合得来。而且,也有更好的人」

「是吗」\\

周露出远望的眼神,心想这是踩到真昼的地雷了。\\

真昼讨厌周看不起自己。据她所说,她不希望周受到不正当的评价。

那么同理,她也讨厌别人贬低周。\\

从周的角度来说,先不说真昼怎么看自己,他在学校并没有展露出原本的模样,因此他不否定不起眼的评价,觉得这个评价很正当。

只不过,真昼会不会允许那就是另一码事了。\\

挂在真昼脸上的笑容一如既往。

但是,她周围的氛围却变得僵硬了一些。\\

「不是,那个」

「哪里不起眼?」

「呃」

「能不能请你具体说说到底哪里不起眼呢?」

「气、气质什么的,长相什么的」

「你会根据长相挑选喜欢的人吗?」

「不、不是,那个」

「你会根据长相来挑选今后可能会相处很久的人吗?」\\

真昼的脸上依然挂着天使般的笑容。明明是这样,她却让人感到莫名的压力。要说原因,恐怕是真昼正有些生气。

就连离那里有一段距离的周都能感到那股压力,与她对峙的当事人想必压力更大。\\

那名男生肯定是发现了真昼面带微笑的怒气。

即使只能看到背影,周也能明白他怕得一动不动。\\

「那、那是……」

「抱歉,我不该那么欺负你」\\

压力消去,真昼的脸庞挂上了困扰一样的柔和笑容。

但是令一向温和、一向面带笑容的真昼生气这个事实,让对峙的那名男生有些踉跄。\\

「让我来订正你的话吧。周君既帅气,又温柔。我还喜欢他稳重、温暖的气质。并且,他还非常绅士,愿意尊重我。在我痛苦的时候,他会在一旁安慰我,为我着想。至少,他不是会说别人坏话、阻碍别人恋爱的人」\\

真昼在最后补上了致命一击。

她宣言的意思也就是「绝对不会喜欢上在我面前说周坏话的你」。\\

「你还有什么要说的吗?」\\

真昼面带可爱的笑容微微歪头催促那名男生说下去。而他似乎已经到极限,以几乎听不见的小声说了一句「不,没有了」,无力地离开了真昼的面前。\\

真昼看向了周,视线毫无阻隔。\\

周困惑着,不知该因为众目睽睽之下几乎等同于告白的话语而脸红,还是该考虑到今后的事情而脸色发青。而真昼朝着他,露出了今天最美的笑容。

那和天使的笑容完全不同,是在家展现出的那种饱含欢喜的甜美笑颜。\\

「周君,一起吃饭吧」

「……嗯」\\

没有男生再去质问周了。\\

\vspace{2\baselineskip}

「结果让她先说了啊」

「唔」\\

从下午比赛开始,经过几种竞技后,终于要轮到骑马战了。周他们正因此集合在一起。听到门胁的低语,周无言以对。\\

他们现在离帐篷很远。之所以这么做,是由于四面八方的目光让人心烦。

尽管现在依然有人注视,但其数量根本无法与近处相比,还算是好的。\\

门胁话中的意思是「应该周先来吧?」因此周无从反驳。\\

「我也有点看出来了。不过椎名和藤宫的关系那么好吗?」\\

九重的神色有些疑惑,似乎是隐隐约约地察觉到周和真昼关系的变化。\\

「嗯,至少从去年开始关系就很好了」

「一直在隐瞒着啊。不过看到今天中午的骚动,倒也可以理解」\\

他向周投来同情遭遇的视线。\\

尽管九重、柊都和周在同一个教室,但在那质问的包围圈中,他们实在无法靠近过来。

两人与周交情不深,他们的判断倒是没错,但周希望树和门胁能稍微帮一下自己。\\

「当时可厉害了。椎名坚决抛下那些没出息的男生的场景,真是看得心情畅快」

「与其说没出息,我觉得只是这事对他们冲击性太大了……」

「嗯,是这样吗?但是,如果是男人,大大方方地对喜欢的女孩子告白就好吧。他们不仅不那样,还一直缠着,甚至说起了藤宫的坏话,够没出息。不敢冒风险却想得到,发现不能到手就胡搅蛮缠,这已经不能说是没出息,根本就是小孩子的行为」

「唔」

「一哉,你的话有一部分戳中藤宫了」\\

「如果是男人就大大方方地告白」这句话,准确地扎进了周的胸口。\\

「嗯,因为我看着藤宫都替他着急」

「那段话就是椎名表明了她的意思吧」\\

这一点周还是知道的。

真昼都已做到这种地步,周也无法再自欺欺人。他可以断言,真昼毫无疑问对自己有好意。\\

(……这一点,我是知道的)\\

周非常清楚,自己胆小窝囊,因为担心被拒绝,所以一直都在逃避。\\

「藤宫喜欢椎名吗?」

「一哉连这都看不出啊……」

「既然这样,向椎名表白不就好了吗?看那个态度,椎名也是喜欢藤宫的吧」

「……我知道。她都迈出这一步了,我也必须得迈出去」\\

周明白,让她做到这种地步,要是自己什么都不做就太丢脸了。

周也明白,因为真昼率直地表达出了好意,所以他应该带着诚意来回答。要回答的内容早已确定,剩下的就是传达方式的问题。\\

柊向下定决心的周投去满足的笑容。\\

「嗯,就是那种感觉。总之就先在骑马战上将对手给冲散吧」\\

柊不知为何很高兴,笑着说「对面毫无疑问会瞄准这边」,对此周只能苦笑。

要骑在上面的九重露出泄气的表情抱怨着「我的负担是不是太大了?」不过那语气听起来与其说是由衷讨厌,不如说是无可奈何,令周稍许放下心来。\\

「藤宫也学学一哉?多冲散点人哦?」

「我尽量」\\

周觉得自己应该拿出男子气概,甩开所有伸向真昼的手,让她成为自己一个人的东西。\\

(总而言之,回到家后好好说清楚吧)\\

为此,下午的比赛要无事度过——周鼓起劲头,而另外三人则相视一笑。
