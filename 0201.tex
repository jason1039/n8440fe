\subsection{想叫上的人}

在周他们的学校,文化节也并非是完全开放的,而是只有家属和朋友可以参加。在此之上还需要提前申请,形式是持票入场,学生申请多少就发多少张票。

当然,每个人能拿到的票是有上限的。\\

之所以采取这一措施,是因为最近日子不太平,以及过去学校里曾经有游客引发过暴力事件。即使是文化节,也要优先考虑学生的安全,为此才有这般决定。\\

「我这边没有人可以叫上」\\

晚饭后,真昼看着学校发下来的申请表,平平淡淡地嘀咕了一声。\\

尽管真昼被称为天使大人,得到众人爱戴,但她基本上不会去结交特定的朋友。初中时期也是如此,没有人能算得上是她亲密无间的朋友。\\

除了朋友,接下来就是父母,而父亲暂且不论,母亲是肯定不会叫来的。说到底,真昼并不愿意叫父母来,因此才做出了没有人可以叫上的结论吧。\\

「我是没有人要好到需要特意叫对方过来,所以这事和我无缘。和我关系好的人都在学校里,这方面不用发愁」

「其实我也是……啊不对,不说的话,我妈他们会很啰嗦……」

「志保子阿姨他们也会参加吗?」

「去年我一直瞒着,结果被啰嗦了好久」\\

志保子发现的时候,闹别扭闹得非常离谱。

从志保子的性格上讲,即使在众人面前,她也肯定会和周亲密接触,而都上高中了还和母亲这么粘是很让人难为情的。除此之外,他也不想让其他人看到父母之间的亲热。\\

今年她是记住了文化节这事,发来消息说『快到文化节了』。这应该是催着要票吧。\\

「我提醒他们一句不要公然亲热,再叫他们过来」

「啊、啊哈哈」\\

真昼也很清楚志保子他们会自然而然地亲热,因此苦笑着。\\

「嗯,要叫的就这两个了吧。老家离这里挺远的,也没人关系好到要叫上对方」

「这样啊」\\

由于了解发生在过去的骚动的一部分,真昼便无意继续说下去了。

周倒是已经不再把这事放在心上,他与升上高中后结交的朋友们也形成了良好的关系,所以这事怎样都无所谓。然而真昼似乎还是很在意。\\

他更在意的,是有父母问题的真昼这边。

真昼的父亲朝阳在人品方面没有问题,但双方都没有要碰面的想法;至于母亲那边,真昼则是压根不想见到。即使只是听到一次她们两个的对话,周也看得出来这一点。真昼绝无可能邀请她来文化祭。\\

即便这样,周也不知道真昼高中前的生活,本以为没法说什么——\\

「……说起来,你说谁都不会叫上,那么管家呢?」

这时周想起来,尽管父母对真昼不理不睬,但有一位女性对她灌注了爱情、给予了教育。\\

真昼做家务的一手本领和厨艺似乎都是那位管家教出来的。讲起那位女性的事情时,真昼也会露出温柔的表情。说那位管家是代替父母把真昼养大的也并不夸张吧。\\

真昼闻言,瞪圆了眼睛。\\

「你还记得小雪啊,我当时应该只是提了几句吧」

「毕竟是你的事情嘛。你不叫那个人吗?」

「……做不到」\\

周觉得自己这主意还不错,然而真昼却神情扭曲,有些寂寞、悲伤,让他注意到自己失言了。\\

「……抱歉」\\

女性管家小雪身上发生了什么变故,他却随口就说把她叫来——周这么想着垂下了眉头,而真昼发现了他在想象什么,慌忙摆手让他不要多想那些。\\

「不是这个意思!我上初中的时候,小雪就不做管家了……她腰不太好」

「……啊」

「虽然这是工作,但毕竟让她一个人管理了这么大的房子,我刚刚那样是感觉辛苦了她,对她过意不去」\\

听到她腰不好,周便觉得,这的确是没办法了。

一旦弄坏了腰,即使治好了也容易复发。这就好比腰上绑着炸弹过日子,既没法工作,也不再能乱来了。\\

「现在她和女儿女婿住在一起。我担心她的身体,所以不好让她过来。学校里没多少地方能方便客人休息,而且她的住址离这里也远,实在是不好意思特地叫她来一趟」

「这样啊,真遗憾」

「嗯」\\

从表情就能看出来,真昼敬慕着那名女管家。

周也想要见一见那个影响到真昼的生活能力和人格形成的人,向她道谢。不过,既然是身体原因,那也没有办法了。\\

「我也有点可惜,好不容易有个照顾你的人,我却没法和她打声招呼。是不是以后去打一声招呼比较好呢」

「咦,打、打招呼?」

「嗯。她就像是你的母亲吧?」

「……是啊」

「那就应该打一声招呼吧」\\

周几乎算是对真昼的亲生父亲做出了收下女儿的宣言,也得到了认同。对于养育她的母亲,也应该说一声才是吧。

从真昼所言来看,真昼受了她很多照顾,她也超出了职务范围去疼爱真昼。对于有这般大恩的人,不讲一声就把女儿带走似乎有些失礼。\\

「那个就过一阵子,等事情确定之后再考虑吧。突然登门拜访也挺没礼貌的,先找个机会写封信……真昼?」

「没、没什么」

「看你表情不像是没什么啊」

「就是没什么」\\

真昼把她喜欢的坐垫摁到周的脸上堵住了他的视野,周便笑道真拿你没办法,由着她这么做了。
