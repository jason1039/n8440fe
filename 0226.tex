\subsection{文化节结束}

「好累啊……」\\

周听着学校里宣布文化节结束的广播,长叹了一口气。\\

父母离店后,同学捉弄得他受了不少罪。不适应的待客工作本就让周绷紧了神经,再加上同学们拿他开涮,精神上积累的疲劳更甚于肉体。

但那些也都迎来了尾声。听到重复播放的广播,周舒展了自己的肩膀。\\

「喂,大家都辛苦了!真的是忙死了」\\

确认客人都走了,也确认了广播后,树哈哈笑着召集起了同学。

看似短却又漫长的文化节迎来结尾,众人的表情中洋溢着成就感,同时也有疲劳之色,后者显然是因为班里的忙碌。\\

「别急着犒劳,要先收拾哦,其实这才是最辛苦的,比准备的时候更费精力。学校通知说,垃圾由学校统一处理,务必尽快把垃圾整理起来」

「噫」

「讨厌啦好麻烦」\\

一说到收拾的事情,同学便一下子没了斗志,产生一片慵懒的气氛。周一边为他们的好懂而苦笑,一边进入收拾的模式,把营业时产生的垃圾塞进袋子,同时倾听着他的声音。\\

「好啦好啦,做完这个就能庆功了。明天可是补休,现在就安心干活吧」

「你也是哦」

「我在干活也有在指示……疼,我知道了别整我了」\\

树在黑板前得意洋洋地挺着胸,有同学戳起了他。树可能也习惯了被摆弄,哈哈笑着参加了收拾的队伍。\\

「庆功会结束后会收相关的费用,别跟我说这两天文化节给花光了啊」

「完了,我还有没有钱来着」

「你自己在名册上写着要参加吧。钱不够的去找人借,或者从我这借钱也行,利息爽到每天一百个点哦」

「这什么高利贷」

「有意见就赶紧收拾,利息给你少收点」

「你也得来干」\\

被同学拍着肩膀的同时,树还高举拳头喊着赶紧做完出去庆祝,给同学打着气。周看着这幅情景露出苦笑,并把大量用剩下的刀叉扔进袋子。和他一样,真昼也边收拾边望着树。\\

「真有精神」

「那家伙就这样」

「庆功会准备在哪办?」

「他说约了卡拉OK的房间,接下来还有\ruby{二次聚会}{家庭餐厅}也可以自由参加」\\

参加庆功会需要事先表明出席的意向,去年周没有加入,但今年不只有树,还有真昼和千岁在,与同学的关系应该也加深了;周有点不太好意思,不过还是准备参加的。

说实话,周不愿在别人面前唱歌,希望能只在旁边听。可是树估计会把麦克风硬塞到他手里,周现在就开始烦恼该如何是好了。\\

「我跟要住在我家的妈妈说好了,稍微晚一些也没事。不过我吧,还是不太喜欢热闹的,只去个卡拉OK就准备回去了」

「我也是这个打算,再说晚饭都准备好了」

「你真能干」

「为了回去后能少费点功夫,这点事情还是要做的」\\

周一边佩服着真昼为晚饭也做好了打算,一边想着「回去后」这几个字暗自笑了出来。接着,真昼疑惑地连连眨眼。

她用目光询问有没有什么有意思的事情,但周没有回答她无言的提问,而是耸耸肩,重新把精神集中到了打扫工作上去。
