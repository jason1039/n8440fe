\subsection{天使大人与新年计划}

圣诞节过去之后,世上就是一片年末的气氛。\\

夜景用的照明灯还留着,然而原本随处可见的圣诞树已经撤下,各种鲜艳的装饰变化成了和风。

贩卖的东西也全面变成了新年装饰和食材,平安夜的样子已经完全不见了踪影。\\

周心想「变化还真是快啊」,一边望着已开始进入过年准备的周遭,一边把脸埋到围巾里取暖。\\

这黑白的千鸟格围巾,是真昼送来的圣诞节礼物。

听真昼说「脖颈处的打扮也很重要」然后便从她那儿收到了这个好东西,它手感非常舒适,挡风保暖性能优良,兼具实用性和装饰性。\\

因为平时不带围巾,于是周便心怀感激地用上了。他确认起手上提着的购物袋里的东西。\\

尽管采购基本是说好分担进行,但是为了减轻要做菜的真昼的负担,一般都是周带着备忘录把东西买齐的。

今天很冷,似乎是要做火锅,袋子里装着蔬菜啊蘑菇啊肉之类的。其中蔬菜偏多,大概是代表了真昼无声的主张「要好好摄入营养」。\\

周再次确认没有遗漏,在越来越冷的天气下哆嗦着,快步回了家。\\

「你回来啦」\\

回到家时已是傍晚,所以真昼前来迎接了周。

无缘的他人迎接房子的主人这个场景尽管有些奇怪,不过周最近已经开始习惯了。\\

「嗯,刚回来……买了点年糕片回来没问题吧?」

「是想涮火锅啊」

「嗯。还有,买了拉面最后吃」

「……我吃不了这么多哦?」

「没关系,我会吃掉一大半的」\\

周以前吃得并不多,不过多亏了真昼的料理,现在晚饭还是吃得挺多的。\\

她可能也是留意着卡路里,吃的量也就是不会发胖的程度。而周吃得比她多,有些微妙的担心,所以最近做起了肌肉锻炼。\\

真昼的感想似乎是说周太瘦了应该多长点肉,所以周希望尽可能多长点肌肉而不是脂肪。\\

「周君吃的话倒是行啦。那个,给我一下,我去放进冰箱。周君去漱口洗手吧」

「好嘞」\\

周把装着货的塑料袋递给真昼之后,老老实实地走去了洗手间。\\

\vspace{2\baselineskip}

「说起来真昼你怎么过年」\\

今天周也是吃完了一如既往非常美味的晚饭。正在收拾打理的时候,他忽然觉得有些在意便向真昼询问。\\

「过年……回去也没意义,就呆在这边」\\

听到这语气实在平淡的回答,周醒悟了自己的失误,而真昼却似乎并不怎么介意。

因为和家人相处不好,所以一提起家庭关系,她必然会摆出冷淡的态度。\\

只是,这样的话,真昼岂不是得一个人过新年吗。

周有大概半年一次要给家人露个脸的约定,没遇到真昼那一会儿,他是准备长假期间回老家的。\\

「周君是要回老家吗」

「这个嘛,姑且家里人有叫我露脸来着」\\

周瞄了一眼真昼。不知是不是错觉,感觉她的眼神比平时的表情要更冷一点。\\

似乎真昼理所当然地以为会自己一个人过,对周会回老家这事并没有怀疑。\\

「……如果回去的话,感觉关于你的事情会被问这问那的啊」

「真辛苦」

「爸爸大概听妈妈说完也就那样了,不过妈妈大概会老想打听吧」

「明明我们经常说话的,真是不可思议呢」

「说真的你不知不觉就跟老妈熟络起来了啊……」\\

不知为何,真昼和老妈不知不觉就打好了关系,结果在周不知道的时候流出了照片和秘密……想着这些,周感到有些虚脱,不过真昼看样子是自愿和老妈交流,并且意外地处得还不错,周觉得这样也还算可以接受。

稍后得叮嘱志保子不要说出多余的事情。这个先不说,周看向了真昼,不知该如何是好。\\

一想到真昼有时会露出的空虚表情和寂寞眼神,周就无论如何……都不想放着她一个人。\\

「我觉得吧,最近也跟妈见过了,虽然对爸有些抱歉,不过这次不回老家应该也可以吧。反正春假会回去的」\\

所以,如果不会给她添麻烦的话,周还是希望能和往常一样和她一起吃晚饭。\\

「……是这样啊」

「嗯。还想吃你的荞麦面\footnote{日本风俗,在除夕夜(12月31日)会吃荞麦面。}来着」

「还真是嘴馋啊」

「因为是真昼做的」

「……明明基本是买来的?」

「就算这样也好啦」\\

就算只是买来的荞麦面煮一煮也好。

因为,两个人慢慢吃面共度时光,这才是更重要的。\\

「……真是个怪人」

「吵死了」\\

对着发表出失礼感想的真昼,周刻意做出了不高兴的回答,而真昼回了个小小的微笑。\\

「……谢谢」

「谢什么啊」

「什么都有」\\

真昼没有再说更多。或许是心情好上了几分,她露出了明朗的表情抱紧了喜爱的坐垫。
