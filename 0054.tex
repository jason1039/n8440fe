\subsection{选择礼物的方法}

周在学习上本就很勤奋,上课态度也很认真,所以并没有太费劲便通过了期末考试。\\

他和真昼一起复查试卷,算出来的分数跟平时差不多,况且平常他在学校里的态度也挺好,留级应该是不用担心了。

树也考了个不错的分数,千岁似乎也避免了挂科,这么一来周熟悉的人就都不用担心留级的问题了。\\

考试之后是毕业典礼,那时会欢送和周没什么关系的高三学生。在毕业典礼后就是休业式了……不过在这之间还有一个节日很成问题。\\

「……该回什么礼呢」\\

这正是情人节的胜者们会迎来的回礼之日。

且先不论周到底算不算是胜者,但既然从真昼和千岁那收了巧克力,他自然是打算要回礼的。\\

只不过,周还在头疼该送什么好。

千岁的话,他打算去那家买了圣诞蛋糕的店准备一个白色情人节款的点心套装,再来点她正在收集的角色周边,就应该差不多了。\\

问题在于真昼。

周总觉得,无论自己送什么她都会欣然收下。

周送的东西她都会平常地收下,而且她看起来比较在意心意,对送的东西并不是很关心。说实话,这样反而让周最头疼。\\

就算想要从喜好的角度来挑选,周也只知道她喜欢甜的东西和可爱的东西,而这种东西女生差不多都喜欢。因而,周一直在头疼到底该选什么好。\\

再怎么说,上回说过的磨刀石肯定是免谈,因为这样不但一点意思也没有,而且预算方面还很紧张。但即使不考虑磨刀石,周还是很烦恼要选什么。如果可以的话,比起实用品而言,周这次更想要送享受用的东西。\\

周暂且先去了杂货店。他望着里面的白色情人节特卖区,但看着这些,周却不太想象得出她真正高兴的样子。

要是这次挑的礼物,能让真昼做出跟上次收到熊布偶时一样的反应就好了。\\

(送两次熊布偶毕竟也没意思啊)\\

虽然可爱的熊布偶货架上倒是摆了一堆,但送两次一样的东西还是欠缺新鲜感吧。

话又说回来,以周那贫乏的想象力,能想到的女生喜欢的东西,除了小饰品以外也没别的了。\\

可周也不敢完全确定,两人的关系到底有没有亲近到能送小饰品。\\

如果送的话,估计真昼还是会好好收下,问题是她到底会不会高兴。

虽然,周觉得两人按男女说来算是关系不错了……但是,送小饰品到底能不能让她高兴呢。\\

这要是树送千岁,那肯定没有问题,但周送真昼就得画一个问号了。\\

周这样一脸烦恼地在特卖区旁边晃来晃去,看上去恐怕就像一个可疑人物吧。

虽然周也换上了外出用装束,但男性在可爱的杂货前晃来晃去肯定还是很可疑。\\

周正念叨着这也不行那也不行,突然有人从身后搭了一句「在找什么吗?」\\

一回头,周便看见一位身着店员围裙的妙龄女性微笑着站在身后。

她大概是看不下去周这实在苦恼的样子,所以才会过来帮忙。要不然的话,她也不会特意向周这正晃来晃去跟个可疑人物似的人搭话。\\

「啊,那个……白色情人节的回礼,我拿不定主意」

「在这边没有中意的吗?别的地方也有些东西常被选来回礼,我带你去看看吧」

「啊,不是这个意思……只是说和她的关系有些不好形容,不知道该送什么不会被讨厌」

「怎么说?」

「她不算是女朋友但挺亲近的……打个比方说,小饰品这种东西,从称不上是喜欢的人那儿收到,会不会感到高兴呢」

因为解释起来很害羞,周的说明便有些含糊,但女店员听完后却笑了起来,恐怕是觉得周的烦恼比较逗人吧。\\

「男性烦恼这种事情是很常见的哦」

「那之前的人都是怎么决定的?」

「大部分人虽然有犹豫,但最后还是会下决心购买。如果关系亲近的话,就算送了,一般也不会被讨厌的」\\

听了「不会讨厌」这话,周稍稍安心了一点。但即使如此,要送真昼饰品还是让他心里有些慌慌的。

真昼虽然身上打理的很整洁,但并不怎么会佩戴饰品。她偶尔倒是会戴一戴,不过每次戴的都是高级的东西。\\

真昼审美的品味很好,因而周并没有自信自己选出的东西能得到她的认可。\\

「有需要的话,到那边去给你介绍几件吧,在女性间很火的」

「……麻烦你了」\\

听见这求之不得的提议,周不自觉地摆正了姿势点了点头。\\

\vspace{2\baselineskip}

「然后我就买了」\\

跟树讲了事情的经过后,周就遭到了笑话。树笑话周的眼神和前几天那个店员一样。\\

两人正在食堂一角吃着每日套餐。一讲到白色情人节的话题,周就不小心把这事讲出来了。\\

「……别多嘴。不过啊,明明没有交往却送对方饰品,还是显得有些恶心对吧」

「你怎么这么矫情啊,男子汉做事得靠勇敢和气势懂不。我觉得那个人啊,只要是你送的,不管收到什么都会开心的哦?」

「……虽然是这么回事啦」\\

以真昼的性格,不论送什么她都会高兴地收下吧。

但周希望送一件能让她真的高兴而且会用的东西,因而还在担心这到底达不达得到要求。\\

「结果你买了个啥?」

「……手链,粉金色、以花为主题的」\\

比起给人以冷淡感的银色和华贵感的金色,感觉还是这华丽中泛着柔和与可爱的粉金色比较适合真昼。\\

身为学生,高价的贵金属肯定是买不起的,所以这里说的只是外观——周自己觉得,在这种颜色的饰品中,他挑选出的这一款设计精致优美,很适合她戴。\\

「咋了,听上去不是挺能让她高兴的嘛」

「……不会被觉得恶心?」

「我说你多虑了吧。为什么你这种地方这么消极……」

「她可是我第一个正式送礼的女性啊」\\

母亲肯定是成不了这样的对象,千岁也不算数。不如说给千岁的东西是她自己闹着要的甜品,周甚至不怎么觉得那算是礼物。\\

「你在这种事情上面还真是缺乏自信啊……」

「不如说怎么可能会有自信……我要送的可是那家伙哦?」

「熊布偶那时候她挺高兴吧」

「虽然是这么回事啦」

「我说啊周,重要的是心意啊心意。既然你钱也花了东西也买了,那剩下的就只有加入你的心意了啊」\\

由于树说得轻巧,周嘟哝着「要是真能那么干脆就好了啊」用手扶住了额。\\

看来,直到白色情人节当天,周都得要在这个决定到底好不好的纠结中度过了。
