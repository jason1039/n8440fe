\subsection{邀请去祭典}

「话说今天有夏祭,要不要去一趟?」\\

在暑假还剩一周的时候,千岁突然午前登门,如是说道。\\

「……我说,这种事情起码提前一天说吧?」\\

她的发言太过突然。要是周有安排的话,她打算怎么办呢。

再说,去夏祭要出门,出门就需准备,应该提前说好才是。\\

幸好周和真昼今天没有安排,晚饭吃什么也还没决定好,可以轻松更改今天的计划。\\

「抱歉抱歉,听阿树说你们挺忙的,我就没好意思联系你们,不小心就到当天了」

「你这么讲的话我也很为难,不过就算这样也该早点说吧?还有,我觉得今天突然过来也很那个啊」

「抱歉了嘛。我姑且跟昼儿联系过哦?」

「不过是到这里十分钟之前联系的……」\\

给千岁准备好冰麦茶后,真昼苦笑着补充道。\\

真昼突然困惑地告诉周「千岁说她要来」,周当然也感到了困惑。树有过突然上门拜访的事迹,可周没想到连千岁都会来这么一着。

千岁大概是确信周在家里才过来的,但周仍然希望她能稍微早点说一声。\\

看到千岁津津有味地喝着凉透的麦茶,周叹了口气,并看了一眼真昼。\\

对于去祭典这事,真昼似乎没有意见。

至于周,考虑到真昼可能由于父亲那件事的影响而情绪略有低落,他也想要带真昼去散散心。尽管真昼的父亲可能还会过来,但周希望他能暂时忘掉这件事。\\

「嗯,要去是可以啦……真昼你觉得呢?要穿浴衣吗?」

「嗯?不了,不巧我手头没有浴衣」

「不是,那个……我家有,估计还是合真昼尺寸的」

「为什么?」

「我妈」\\

周刚一提醒志保子的存在,真昼便恍然大悟。在真昼的心里,志保子是个只要有可爱的衣服就想往真昼身上套的人吧,而这一点并没有错,让人笑都笑不出来。\\

前些日子从老家回来的时候,在寄来的行李中明显有好几件不该由周穿的衣服。\\

『有机会的话就让真昼穿,照片也拜托啦』\\

这样一条便签和浴衣什么的塞在一起,周还记得自己当时的无语。\\

「咦,昼儿要穿浴衣?我要看~!」

「你不穿吗」

「我才不要。虽然可爱但是不方便运动,而且系着衣带可能还吃不饱~」

「那只是你贪吃吧」

「你很没礼貌哎」\\

千岁不太喜欢拘束的打扮,而且还是吃得多动得多的类型,所以不想穿浴衣这种需要文雅的衣服。\\

「说起来,树呢?」

「嗯?阿树会来的,计划到现场汇合」

「你说的好像确定了我们要去一样……」

「呵呵,我想着昼儿也不会拒绝嘛」

「你倒是考虑下我们方不方便啊……虽然确实没什么事」

「抱歉抱歉」\\

周朝着没在反省的千岁眯起眼睛,这也是没办法的吧。

毕竟周给树发过消息说这几天没事情,会决定邀请应该是因为这个。

周希望千岁能先约好再过来,不过转换心情也很重要,这次千岁的邀请让人感激。\\

「所以,真昼怎么说?想穿浴衣吗?」

「……只有我穿浴衣的话,不会太显眼吗?」

「要是真昼讨厌自己一个人穿,我也能穿啦……」

「咦,周君也有吗」

「我妈很精明地给放在一起了」\\

志保子大概是关照周穿着这些去祭典吧。由于真昼父亲那件事,周完全忘记查有没有夏祭了,这么一想,千岁邀请的时机或许恰到好处。\\

真昼听到周也穿浴衣,便一下子显得心神不定起来,而周则在内心自言自语说看男人穿浴衣也没什么可高兴的。\\

这不是周的自卑:女性穿浴衣很华美,但男性并非如此;虽然能出氛围,但周觉得不至于到可以欣赏的地步。\\

不过,真昼瞄着他,就像在说想看似的。可爱的女朋友那么想要,要穿浴衣也没问题。要走在真昼身边,穿上浴衣也会更美观一些吧。\\

「真昼想看的话我就穿吧」

「我、我想看」

「回答好快啊。行吧,不过你不用太期待,我的浴衣很普通的」\\

浴衣是深蓝色的,没有花纹,加上红豆色的腰带,配色低调,不会显眼夺目。\\

尽管如此,真昼却报以期待的眼神。于是,周也苦笑道「我会尽可能穿得合身的」摸起真昼的头。
